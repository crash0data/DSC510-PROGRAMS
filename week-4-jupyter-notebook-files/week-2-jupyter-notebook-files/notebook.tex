
% Default to the notebook output style

    


% Inherit from the specified cell style.




    
\documentclass[11pt]{article}

    
    
    \usepackage[T1]{fontenc}
    % Nicer default font (+ math font) than Computer Modern for most use cases
    \usepackage{mathpazo}

    % Basic figure setup, for now with no caption control since it's done
    % automatically by Pandoc (which extracts ![](path) syntax from Markdown).
    \usepackage{graphicx}
    % We will generate all images so they have a width \maxwidth. This means
    % that they will get their normal width if they fit onto the page, but
    % are scaled down if they would overflow the margins.
    \makeatletter
    \def\maxwidth{\ifdim\Gin@nat@width>\linewidth\linewidth
    \else\Gin@nat@width\fi}
    \makeatother
    \let\Oldincludegraphics\includegraphics
    % Set max figure width to be 80% of text width, for now hardcoded.
    \renewcommand{\includegraphics}[1]{\Oldincludegraphics[width=.8\maxwidth]{#1}}
    % Ensure that by default, figures have no caption (until we provide a
    % proper Figure object with a Caption API and a way to capture that
    % in the conversion process - todo).
    \usepackage{caption}
    \DeclareCaptionLabelFormat{nolabel}{}
    \captionsetup{labelformat=nolabel}

    \usepackage{adjustbox} % Used to constrain images to a maximum size 
    \usepackage{xcolor} % Allow colors to be defined
    \usepackage{enumerate} % Needed for markdown enumerations to work
    \usepackage{geometry} % Used to adjust the document margins
    \usepackage{amsmath} % Equations
    \usepackage{amssymb} % Equations
    \usepackage{textcomp} % defines textquotesingle
    % Hack from http://tex.stackexchange.com/a/47451/13684:
    \AtBeginDocument{%
        \def\PYZsq{\textquotesingle}% Upright quotes in Pygmentized code
    }
    \usepackage{upquote} % Upright quotes for verbatim code
    \usepackage{eurosym} % defines \euro
    \usepackage[mathletters]{ucs} % Extended unicode (utf-8) support
    \usepackage[utf8x]{inputenc} % Allow utf-8 characters in the tex document
    \usepackage{fancyvrb} % verbatim replacement that allows latex
    \usepackage{grffile} % extends the file name processing of package graphics 
                         % to support a larger range 
    % The hyperref package gives us a pdf with properly built
    % internal navigation ('pdf bookmarks' for the table of contents,
    % internal cross-reference links, web links for URLs, etc.)
    \usepackage{hyperref}
    \usepackage{longtable} % longtable support required by pandoc >1.10
    \usepackage{booktabs}  % table support for pandoc > 1.12.2
    \usepackage[inline]{enumitem} % IRkernel/repr support (it uses the enumerate* environment)
    \usepackage[normalem]{ulem} % ulem is needed to support strikethroughs (\sout)
                                % normalem makes italics be italics, not underlines
    

    
    
    % Colors for the hyperref package
    \definecolor{urlcolor}{rgb}{0,.145,.698}
    \definecolor{linkcolor}{rgb}{.71,0.21,0.01}
    \definecolor{citecolor}{rgb}{.12,.54,.11}

    % ANSI colors
    \definecolor{ansi-black}{HTML}{3E424D}
    \definecolor{ansi-black-intense}{HTML}{282C36}
    \definecolor{ansi-red}{HTML}{E75C58}
    \definecolor{ansi-red-intense}{HTML}{B22B31}
    \definecolor{ansi-green}{HTML}{00A250}
    \definecolor{ansi-green-intense}{HTML}{007427}
    \definecolor{ansi-yellow}{HTML}{DDB62B}
    \definecolor{ansi-yellow-intense}{HTML}{B27D12}
    \definecolor{ansi-blue}{HTML}{208FFB}
    \definecolor{ansi-blue-intense}{HTML}{0065CA}
    \definecolor{ansi-magenta}{HTML}{D160C4}
    \definecolor{ansi-magenta-intense}{HTML}{A03196}
    \definecolor{ansi-cyan}{HTML}{60C6C8}
    \definecolor{ansi-cyan-intense}{HTML}{258F8F}
    \definecolor{ansi-white}{HTML}{C5C1B4}
    \definecolor{ansi-white-intense}{HTML}{A1A6B2}

    % commands and environments needed by pandoc snippets
    % extracted from the output of `pandoc -s`
    \providecommand{\tightlist}{%
      \setlength{\itemsep}{0pt}\setlength{\parskip}{0pt}}
    \DefineVerbatimEnvironment{Highlighting}{Verbatim}{commandchars=\\\{\}}
    % Add ',fontsize=\small' for more characters per line
    \newenvironment{Shaded}{}{}
    \newcommand{\KeywordTok}[1]{\textcolor[rgb]{0.00,0.44,0.13}{\textbf{{#1}}}}
    \newcommand{\DataTypeTok}[1]{\textcolor[rgb]{0.56,0.13,0.00}{{#1}}}
    \newcommand{\DecValTok}[1]{\textcolor[rgb]{0.25,0.63,0.44}{{#1}}}
    \newcommand{\BaseNTok}[1]{\textcolor[rgb]{0.25,0.63,0.44}{{#1}}}
    \newcommand{\FloatTok}[1]{\textcolor[rgb]{0.25,0.63,0.44}{{#1}}}
    \newcommand{\CharTok}[1]{\textcolor[rgb]{0.25,0.44,0.63}{{#1}}}
    \newcommand{\StringTok}[1]{\textcolor[rgb]{0.25,0.44,0.63}{{#1}}}
    \newcommand{\CommentTok}[1]{\textcolor[rgb]{0.38,0.63,0.69}{\textit{{#1}}}}
    \newcommand{\OtherTok}[1]{\textcolor[rgb]{0.00,0.44,0.13}{{#1}}}
    \newcommand{\AlertTok}[1]{\textcolor[rgb]{1.00,0.00,0.00}{\textbf{{#1}}}}
    \newcommand{\FunctionTok}[1]{\textcolor[rgb]{0.02,0.16,0.49}{{#1}}}
    \newcommand{\RegionMarkerTok}[1]{{#1}}
    \newcommand{\ErrorTok}[1]{\textcolor[rgb]{1.00,0.00,0.00}{\textbf{{#1}}}}
    \newcommand{\NormalTok}[1]{{#1}}
    
    % Additional commands for more recent versions of Pandoc
    \newcommand{\ConstantTok}[1]{\textcolor[rgb]{0.53,0.00,0.00}{{#1}}}
    \newcommand{\SpecialCharTok}[1]{\textcolor[rgb]{0.25,0.44,0.63}{{#1}}}
    \newcommand{\VerbatimStringTok}[1]{\textcolor[rgb]{0.25,0.44,0.63}{{#1}}}
    \newcommand{\SpecialStringTok}[1]{\textcolor[rgb]{0.73,0.40,0.53}{{#1}}}
    \newcommand{\ImportTok}[1]{{#1}}
    \newcommand{\DocumentationTok}[1]{\textcolor[rgb]{0.73,0.13,0.13}{\textit{{#1}}}}
    \newcommand{\AnnotationTok}[1]{\textcolor[rgb]{0.38,0.63,0.69}{\textbf{\textit{{#1}}}}}
    \newcommand{\CommentVarTok}[1]{\textcolor[rgb]{0.38,0.63,0.69}{\textbf{\textit{{#1}}}}}
    \newcommand{\VariableTok}[1]{\textcolor[rgb]{0.10,0.09,0.49}{{#1}}}
    \newcommand{\ControlFlowTok}[1]{\textcolor[rgb]{0.00,0.44,0.13}{\textbf{{#1}}}}
    \newcommand{\OperatorTok}[1]{\textcolor[rgb]{0.40,0.40,0.40}{{#1}}}
    \newcommand{\BuiltInTok}[1]{{#1}}
    \newcommand{\ExtensionTok}[1]{{#1}}
    \newcommand{\PreprocessorTok}[1]{\textcolor[rgb]{0.74,0.48,0.00}{{#1}}}
    \newcommand{\AttributeTok}[1]{\textcolor[rgb]{0.49,0.56,0.16}{{#1}}}
    \newcommand{\InformationTok}[1]{\textcolor[rgb]{0.38,0.63,0.69}{\textbf{\textit{{#1}}}}}
    \newcommand{\WarningTok}[1]{\textcolor[rgb]{0.38,0.63,0.69}{\textbf{\textit{{#1}}}}}
    
    
    % Define a nice break command that doesn't care if a line doesn't already
    % exist.
    \def\br{\hspace*{\fill} \\* }
    % Math Jax compatability definitions
    \def\gt{>}
    \def\lt{<}
    % Document parameters
    \title{CHERRERA-510 Assignment 2}
    
    
    

    % Pygments definitions
    
\makeatletter
\def\PY@reset{\let\PY@it=\relax \let\PY@bf=\relax%
    \let\PY@ul=\relax \let\PY@tc=\relax%
    \let\PY@bc=\relax \let\PY@ff=\relax}
\def\PY@tok#1{\csname PY@tok@#1\endcsname}
\def\PY@toks#1+{\ifx\relax#1\empty\else%
    \PY@tok{#1}\expandafter\PY@toks\fi}
\def\PY@do#1{\PY@bc{\PY@tc{\PY@ul{%
    \PY@it{\PY@bf{\PY@ff{#1}}}}}}}
\def\PY#1#2{\PY@reset\PY@toks#1+\relax+\PY@do{#2}}

\expandafter\def\csname PY@tok@w\endcsname{\def\PY@tc##1{\textcolor[rgb]{0.73,0.73,0.73}{##1}}}
\expandafter\def\csname PY@tok@c\endcsname{\let\PY@it=\textit\def\PY@tc##1{\textcolor[rgb]{0.25,0.50,0.50}{##1}}}
\expandafter\def\csname PY@tok@cp\endcsname{\def\PY@tc##1{\textcolor[rgb]{0.74,0.48,0.00}{##1}}}
\expandafter\def\csname PY@tok@k\endcsname{\let\PY@bf=\textbf\def\PY@tc##1{\textcolor[rgb]{0.00,0.50,0.00}{##1}}}
\expandafter\def\csname PY@tok@kp\endcsname{\def\PY@tc##1{\textcolor[rgb]{0.00,0.50,0.00}{##1}}}
\expandafter\def\csname PY@tok@kt\endcsname{\def\PY@tc##1{\textcolor[rgb]{0.69,0.00,0.25}{##1}}}
\expandafter\def\csname PY@tok@o\endcsname{\def\PY@tc##1{\textcolor[rgb]{0.40,0.40,0.40}{##1}}}
\expandafter\def\csname PY@tok@ow\endcsname{\let\PY@bf=\textbf\def\PY@tc##1{\textcolor[rgb]{0.67,0.13,1.00}{##1}}}
\expandafter\def\csname PY@tok@nb\endcsname{\def\PY@tc##1{\textcolor[rgb]{0.00,0.50,0.00}{##1}}}
\expandafter\def\csname PY@tok@nf\endcsname{\def\PY@tc##1{\textcolor[rgb]{0.00,0.00,1.00}{##1}}}
\expandafter\def\csname PY@tok@nc\endcsname{\let\PY@bf=\textbf\def\PY@tc##1{\textcolor[rgb]{0.00,0.00,1.00}{##1}}}
\expandafter\def\csname PY@tok@nn\endcsname{\let\PY@bf=\textbf\def\PY@tc##1{\textcolor[rgb]{0.00,0.00,1.00}{##1}}}
\expandafter\def\csname PY@tok@ne\endcsname{\let\PY@bf=\textbf\def\PY@tc##1{\textcolor[rgb]{0.82,0.25,0.23}{##1}}}
\expandafter\def\csname PY@tok@nv\endcsname{\def\PY@tc##1{\textcolor[rgb]{0.10,0.09,0.49}{##1}}}
\expandafter\def\csname PY@tok@no\endcsname{\def\PY@tc##1{\textcolor[rgb]{0.53,0.00,0.00}{##1}}}
\expandafter\def\csname PY@tok@nl\endcsname{\def\PY@tc##1{\textcolor[rgb]{0.63,0.63,0.00}{##1}}}
\expandafter\def\csname PY@tok@ni\endcsname{\let\PY@bf=\textbf\def\PY@tc##1{\textcolor[rgb]{0.60,0.60,0.60}{##1}}}
\expandafter\def\csname PY@tok@na\endcsname{\def\PY@tc##1{\textcolor[rgb]{0.49,0.56,0.16}{##1}}}
\expandafter\def\csname PY@tok@nt\endcsname{\let\PY@bf=\textbf\def\PY@tc##1{\textcolor[rgb]{0.00,0.50,0.00}{##1}}}
\expandafter\def\csname PY@tok@nd\endcsname{\def\PY@tc##1{\textcolor[rgb]{0.67,0.13,1.00}{##1}}}
\expandafter\def\csname PY@tok@s\endcsname{\def\PY@tc##1{\textcolor[rgb]{0.73,0.13,0.13}{##1}}}
\expandafter\def\csname PY@tok@sd\endcsname{\let\PY@it=\textit\def\PY@tc##1{\textcolor[rgb]{0.73,0.13,0.13}{##1}}}
\expandafter\def\csname PY@tok@si\endcsname{\let\PY@bf=\textbf\def\PY@tc##1{\textcolor[rgb]{0.73,0.40,0.53}{##1}}}
\expandafter\def\csname PY@tok@se\endcsname{\let\PY@bf=\textbf\def\PY@tc##1{\textcolor[rgb]{0.73,0.40,0.13}{##1}}}
\expandafter\def\csname PY@tok@sr\endcsname{\def\PY@tc##1{\textcolor[rgb]{0.73,0.40,0.53}{##1}}}
\expandafter\def\csname PY@tok@ss\endcsname{\def\PY@tc##1{\textcolor[rgb]{0.10,0.09,0.49}{##1}}}
\expandafter\def\csname PY@tok@sx\endcsname{\def\PY@tc##1{\textcolor[rgb]{0.00,0.50,0.00}{##1}}}
\expandafter\def\csname PY@tok@m\endcsname{\def\PY@tc##1{\textcolor[rgb]{0.40,0.40,0.40}{##1}}}
\expandafter\def\csname PY@tok@gh\endcsname{\let\PY@bf=\textbf\def\PY@tc##1{\textcolor[rgb]{0.00,0.00,0.50}{##1}}}
\expandafter\def\csname PY@tok@gu\endcsname{\let\PY@bf=\textbf\def\PY@tc##1{\textcolor[rgb]{0.50,0.00,0.50}{##1}}}
\expandafter\def\csname PY@tok@gd\endcsname{\def\PY@tc##1{\textcolor[rgb]{0.63,0.00,0.00}{##1}}}
\expandafter\def\csname PY@tok@gi\endcsname{\def\PY@tc##1{\textcolor[rgb]{0.00,0.63,0.00}{##1}}}
\expandafter\def\csname PY@tok@gr\endcsname{\def\PY@tc##1{\textcolor[rgb]{1.00,0.00,0.00}{##1}}}
\expandafter\def\csname PY@tok@ge\endcsname{\let\PY@it=\textit}
\expandafter\def\csname PY@tok@gs\endcsname{\let\PY@bf=\textbf}
\expandafter\def\csname PY@tok@gp\endcsname{\let\PY@bf=\textbf\def\PY@tc##1{\textcolor[rgb]{0.00,0.00,0.50}{##1}}}
\expandafter\def\csname PY@tok@go\endcsname{\def\PY@tc##1{\textcolor[rgb]{0.53,0.53,0.53}{##1}}}
\expandafter\def\csname PY@tok@gt\endcsname{\def\PY@tc##1{\textcolor[rgb]{0.00,0.27,0.87}{##1}}}
\expandafter\def\csname PY@tok@err\endcsname{\def\PY@bc##1{\setlength{\fboxsep}{0pt}\fcolorbox[rgb]{1.00,0.00,0.00}{1,1,1}{\strut ##1}}}
\expandafter\def\csname PY@tok@kc\endcsname{\let\PY@bf=\textbf\def\PY@tc##1{\textcolor[rgb]{0.00,0.50,0.00}{##1}}}
\expandafter\def\csname PY@tok@kd\endcsname{\let\PY@bf=\textbf\def\PY@tc##1{\textcolor[rgb]{0.00,0.50,0.00}{##1}}}
\expandafter\def\csname PY@tok@kn\endcsname{\let\PY@bf=\textbf\def\PY@tc##1{\textcolor[rgb]{0.00,0.50,0.00}{##1}}}
\expandafter\def\csname PY@tok@kr\endcsname{\let\PY@bf=\textbf\def\PY@tc##1{\textcolor[rgb]{0.00,0.50,0.00}{##1}}}
\expandafter\def\csname PY@tok@bp\endcsname{\def\PY@tc##1{\textcolor[rgb]{0.00,0.50,0.00}{##1}}}
\expandafter\def\csname PY@tok@fm\endcsname{\def\PY@tc##1{\textcolor[rgb]{0.00,0.00,1.00}{##1}}}
\expandafter\def\csname PY@tok@vc\endcsname{\def\PY@tc##1{\textcolor[rgb]{0.10,0.09,0.49}{##1}}}
\expandafter\def\csname PY@tok@vg\endcsname{\def\PY@tc##1{\textcolor[rgb]{0.10,0.09,0.49}{##1}}}
\expandafter\def\csname PY@tok@vi\endcsname{\def\PY@tc##1{\textcolor[rgb]{0.10,0.09,0.49}{##1}}}
\expandafter\def\csname PY@tok@vm\endcsname{\def\PY@tc##1{\textcolor[rgb]{0.10,0.09,0.49}{##1}}}
\expandafter\def\csname PY@tok@sa\endcsname{\def\PY@tc##1{\textcolor[rgb]{0.73,0.13,0.13}{##1}}}
\expandafter\def\csname PY@tok@sb\endcsname{\def\PY@tc##1{\textcolor[rgb]{0.73,0.13,0.13}{##1}}}
\expandafter\def\csname PY@tok@sc\endcsname{\def\PY@tc##1{\textcolor[rgb]{0.73,0.13,0.13}{##1}}}
\expandafter\def\csname PY@tok@dl\endcsname{\def\PY@tc##1{\textcolor[rgb]{0.73,0.13,0.13}{##1}}}
\expandafter\def\csname PY@tok@s2\endcsname{\def\PY@tc##1{\textcolor[rgb]{0.73,0.13,0.13}{##1}}}
\expandafter\def\csname PY@tok@sh\endcsname{\def\PY@tc##1{\textcolor[rgb]{0.73,0.13,0.13}{##1}}}
\expandafter\def\csname PY@tok@s1\endcsname{\def\PY@tc##1{\textcolor[rgb]{0.73,0.13,0.13}{##1}}}
\expandafter\def\csname PY@tok@mb\endcsname{\def\PY@tc##1{\textcolor[rgb]{0.40,0.40,0.40}{##1}}}
\expandafter\def\csname PY@tok@mf\endcsname{\def\PY@tc##1{\textcolor[rgb]{0.40,0.40,0.40}{##1}}}
\expandafter\def\csname PY@tok@mh\endcsname{\def\PY@tc##1{\textcolor[rgb]{0.40,0.40,0.40}{##1}}}
\expandafter\def\csname PY@tok@mi\endcsname{\def\PY@tc##1{\textcolor[rgb]{0.40,0.40,0.40}{##1}}}
\expandafter\def\csname PY@tok@il\endcsname{\def\PY@tc##1{\textcolor[rgb]{0.40,0.40,0.40}{##1}}}
\expandafter\def\csname PY@tok@mo\endcsname{\def\PY@tc##1{\textcolor[rgb]{0.40,0.40,0.40}{##1}}}
\expandafter\def\csname PY@tok@ch\endcsname{\let\PY@it=\textit\def\PY@tc##1{\textcolor[rgb]{0.25,0.50,0.50}{##1}}}
\expandafter\def\csname PY@tok@cm\endcsname{\let\PY@it=\textit\def\PY@tc##1{\textcolor[rgb]{0.25,0.50,0.50}{##1}}}
\expandafter\def\csname PY@tok@cpf\endcsname{\let\PY@it=\textit\def\PY@tc##1{\textcolor[rgb]{0.25,0.50,0.50}{##1}}}
\expandafter\def\csname PY@tok@c1\endcsname{\let\PY@it=\textit\def\PY@tc##1{\textcolor[rgb]{0.25,0.50,0.50}{##1}}}
\expandafter\def\csname PY@tok@cs\endcsname{\let\PY@it=\textit\def\PY@tc##1{\textcolor[rgb]{0.25,0.50,0.50}{##1}}}

\def\PYZbs{\char`\\}
\def\PYZus{\char`\_}
\def\PYZob{\char`\{}
\def\PYZcb{\char`\}}
\def\PYZca{\char`\^}
\def\PYZam{\char`\&}
\def\PYZlt{\char`\<}
\def\PYZgt{\char`\>}
\def\PYZsh{\char`\#}
\def\PYZpc{\char`\%}
\def\PYZdl{\char`\$}
\def\PYZhy{\char`\-}
\def\PYZsq{\char`\'}
\def\PYZdq{\char`\"}
\def\PYZti{\char`\~}
% for compatibility with earlier versions
\def\PYZat{@}
\def\PYZlb{[}
\def\PYZrb{]}
\makeatother


    % Exact colors from NB
    \definecolor{incolor}{rgb}{0.0, 0.0, 0.5}
    \definecolor{outcolor}{rgb}{0.545, 0.0, 0.0}



    
    % Prevent overflowing lines due to hard-to-break entities
    \sloppy 
    % Setup hyperref package
    \hypersetup{
      breaklinks=true,  % so long urls are correctly broken across lines
      colorlinks=true,
      urlcolor=urlcolor,
      linkcolor=linkcolor,
      citecolor=citecolor,
      }
    % Slightly bigger margins than the latex defaults
    
    \geometry{verbose,tmargin=1in,bmargin=1in,lmargin=1in,rmargin=1in}
    
    

    \begin{document}
    
    
    \maketitle
    
    

    
    \hypertarget{assignment-2---simple-types}{%
\section{Assignment 2 - Simple
Types}\label{assignment-2---simple-types}}

\hypertarget{instructions}{%
\subsection{Instructions}\label{instructions}}

Follow the instructions for submitting a Jupyter Notebook assignment in
the submitting assignments documentation.

    \hypertarget{numeric-operations-20-points}{%
\subsection{1. Numeric Operations (20
Points)}\label{numeric-operations-20-points}}

\begin{Shaded}
\begin{Highlighting}[]
\NormalTok{a }\OperatorTok{=} \DecValTok{402}
\NormalTok{b }\OperatorTok{=} \DecValTok{1855}

\NormalTok{x }\OperatorTok{=} \FloatTok{41.151309}
\NormalTok{y }\OperatorTok{=} \FloatTok{-95.919741}
\end{Highlighting}
\end{Shaded}

Given the preceding variable definitions, answer complete the questions
below.

\begin{enumerate}
\def\labelenumi{\alph{enumi}.}
\item
  Compute the absolute value of \texttt{y}
\item
  Add \texttt{x} and \texttt{y} and multiple the result by \texttt{a}
\item
  Calculate the remainder leftover after dividing \texttt{b} by
  \texttt{a} (i.e. \texttt{b/a})
\item
  Calculate \texttt{a} to the power of \texttt{3}
\item
  Show how to convert \texttt{a} to a floating point number
\item
  Multiple \texttt{x} by \texttt{y} and round the result to two
  signficant digits
\item
  Compute the bitwise \emph{or} of \texttt{a} and \texttt{b}
\item
  Compute \texttt{x} divided by negative \texttt{y}
\item
  Compute \texttt{a} added to \texttt{b} divided by \texttt{x} minus
  \texttt{y}
\item
  Compute the floored quotient of \texttt{b} and \texttt{x}
\end{enumerate}

    \begin{Verbatim}[commandchars=\\\{\}]
{\color{incolor}In [{\color{incolor}4}]:} \PY{c+c1}{\PYZsh{} a. Compute the absolute value of y}
        
        \PY{n}{floating} \PY{o}{=} \PY{o}{\PYZhy{}}\PY{l+m+mf}{95.919741}
        \PY{n+nb}{print}\PY{p}{(}\PY{l+s+s1}{\PYZsq{}}\PY{l+s+s1}{Absolute value of \PYZhy{}95.919741 is:}\PY{l+s+s1}{\PYZsq{}}\PY{p}{,} \PY{n+nb}{abs}\PY{p}{(}\PY{n}{floating}\PY{p}{)}\PY{p}{)}
\end{Verbatim}


    \begin{Verbatim}[commandchars=\\\{\}]
Absolute value of -95.919741 is: 95.919741

    \end{Verbatim}

    \begin{Verbatim}[commandchars=\\\{\}]
{\color{incolor}In [{\color{incolor}5}]:} \PY{c+c1}{\PYZsh{} b. Add x and y and multiple the result by a}
        \PY{n}{a} \PY{o}{=} \PY{l+m+mi}{402}
        \PY{n}{x} \PY{o}{=} \PY{l+m+mf}{41.151309}
        \PY{n}{y} \PY{o}{=} \PY{o}{\PYZhy{}}\PY{l+m+mf}{95.919741}
        
        \PY{n+nb}{print} \PY{p}{(}\PY{p}{(}\PY{n}{x}\PY{o}{+}\PY{n}{y}\PY{p}{)}\PY{o}{*}\PY{n}{a}\PY{p}{)}
\end{Verbatim}


    \begin{Verbatim}[commandchars=\\\{\}]
-22016.909664000003

    \end{Verbatim}

    \begin{Verbatim}[commandchars=\\\{\}]
{\color{incolor}In [{\color{incolor}9}]:} \PY{c+c1}{\PYZsh{} c. Calculate the remainder leftover after dividing b by a (i.e. b/a)}
        \PY{n}{a} \PY{o}{=} \PY{l+m+mi}{402}
        \PY{n}{b} \PY{o}{=} \PY{l+m+mi}{1855}
        
        \PY{n+nb}{print}\PY{p}{(}\PY{n}{b}\PY{o}{/}\PY{n}{a}\PY{p}{)} \PY{c+c1}{\PYZsh{} will return a floating point number}
        \PY{n+nb}{print}\PY{p}{(}\PY{n}{b}\PY{o}{/}\PY{o}{/}\PY{n}{a}\PY{p}{)} \PY{c+c1}{\PYZsh{} returns an integer }
\end{Verbatim}


    \begin{Verbatim}[commandchars=\\\{\}]
4.614427860696518
4

    \end{Verbatim}

    \begin{Verbatim}[commandchars=\\\{\}]
{\color{incolor}In [{\color{incolor}10}]:} \PY{c+c1}{\PYZsh{} d. Calculate a to the power of 3}
         \PY{n}{a} \PY{o}{=} \PY{l+m+mi}{402}
         \PY{n+nb}{print}\PY{p}{(}\PY{n}{a}\PY{o}{*}\PY{o}{*}\PY{l+m+mi}{3}\PY{p}{)}
\end{Verbatim}


    \begin{Verbatim}[commandchars=\\\{\}]
64964808

    \end{Verbatim}

    \begin{Verbatim}[commandchars=\\\{\}]
{\color{incolor}In [{\color{incolor}13}]:} \PY{c+c1}{\PYZsh{} e. Show how to convert a to a floating point number}
         \PY{n}{a} \PY{o}{=} \PY{l+m+mi}{402}
         \PY{n+nb}{print}\PY{p}{(}\PY{n+nb}{type}\PY{p}{(}\PY{n}{a}\PY{p}{)}\PY{p}{)} \PY{c+c1}{\PYZsh{}verify data type}
         \PY{n+nb}{print}\PY{p}{(}\PY{n}{a}\PY{o}{/}\PY{l+m+mi}{1}\PY{p}{)} \PY{c+c1}{\PYZsh{} if you divide the integer by 1 it will convert it to floating point number }
\end{Verbatim}


    \begin{Verbatim}[commandchars=\\\{\}]
<class 'int'>
402.0

    \end{Verbatim}

    \begin{Verbatim}[commandchars=\\\{\}]
{\color{incolor}In [{\color{incolor}19}]:} \PY{c+c1}{\PYZsh{} f. Multiple x by y and round the result to two signficant digits}
         \PY{n}{x} \PY{o}{=} \PY{l+m+mf}{41.151309}
         \PY{n}{y} \PY{o}{=} \PY{o}{\PYZhy{}}\PY{l+m+mf}{95.919741}
         
         \PY{n}{total} \PY{o}{=} \PY{n}{x} \PY{o}{*} \PY{n}{y}
         
         \PY{n+nb}{print} \PY{p}{(}\PY{n}{total}\PY{p}{)}
         \PY{n+nb}{print}\PY{p}{(}\PY{n+nb}{round}\PY{p}{(}\PY{n}{total}\PY{p}{,}\PY{l+m+mi}{2}\PY{p}{)}\PY{p}{)}
\end{Verbatim}


    \begin{Verbatim}[commandchars=\\\{\}]
-3947.2229010909687
-3947.22

    \end{Verbatim}

    \begin{Verbatim}[commandchars=\\\{\}]
{\color{incolor}In [{\color{incolor}20}]:} \PY{c+c1}{\PYZsh{} g. Compute the bitwise or of a and b}
         \PY{n}{a} \PY{o}{=} \PY{l+m+mi}{402}
         \PY{n}{b} \PY{o}{=} \PY{l+m+mi}{1855}
         \PY{n+nb}{print}\PY{p}{(}\PY{n}{a} \PY{o}{|} \PY{n}{b}\PY{p}{)}  \PY{c+c1}{\PYZsh{} compute bitwise OR, which is set union}
\end{Verbatim}


    \begin{Verbatim}[commandchars=\\\{\}]
1983

    \end{Verbatim}

    \begin{Verbatim}[commandchars=\\\{\}]
{\color{incolor}In [{\color{incolor}22}]:} \PY{c+c1}{\PYZsh{} h. Compute x divided by negative y}
         
         \PY{n}{x} \PY{o}{=} \PY{l+m+mf}{41.151309}
         \PY{n}{y} \PY{o}{=} \PY{o}{\PYZhy{}}\PY{l+m+mf}{95.919741}
         
         \PY{n+nb}{print}\PY{p}{(}\PY{n}{x} \PY{o}{/} \PY{o}{\PYZhy{}}\PY{n}{y}\PY{p}{)} \PY{c+c1}{\PYZsh{} by dividing by negative y it returns a positive number}
\end{Verbatim}


    \begin{Verbatim}[commandchars=\\\{\}]
0.4290181413229629

    \end{Verbatim}

    \begin{Verbatim}[commandchars=\\\{\}]
{\color{incolor}In [{\color{incolor}23}]:} \PY{c+c1}{\PYZsh{} i. Compute a added to b divided by x minus y}
         \PY{n}{a} \PY{o}{=} \PY{l+m+mi}{402}
         \PY{n}{b} \PY{o}{=} \PY{l+m+mi}{1855}
         
         \PY{n}{x} \PY{o}{=} \PY{l+m+mf}{41.151309}
         \PY{n}{y} \PY{o}{=} \PY{o}{\PYZhy{}}\PY{l+m+mf}{95.919741}
         
         \PY{n+nb}{print}\PY{p}{(}\PY{n}{a} \PY{o}{+} \PY{n}{b} \PY{o}{/} \PY{n}{x} \PY{o}{\PYZhy{}} \PY{n}{y}\PY{p}{)}
\end{Verbatim}


    \begin{Verbatim}[commandchars=\\\{\}]
542.9972864068789

    \end{Verbatim}

    \begin{Verbatim}[commandchars=\\\{\}]
{\color{incolor}In [{\color{incolor}25}]:} \PY{c+c1}{\PYZsh{} j. Compute the floored quotient of b and x}
         \PY{n}{a} \PY{o}{=} \PY{l+m+mi}{402}
         \PY{n}{b} \PY{o}{=} \PY{l+m+mi}{1855}
         
         \PY{n}{x} \PY{o}{=} \PY{l+m+mf}{41.151309}
         \PY{n}{y} \PY{o}{=} \PY{o}{\PYZhy{}}\PY{l+m+mf}{95.919741}
         
         \PY{n+nb}{print}\PY{p}{(}\PY{n}{b}\PY{o}{/}\PY{o}{/}\PY{n}{x}\PY{p}{)}
\end{Verbatim}


    \begin{Verbatim}[commandchars=\\\{\}]
45.0

    \end{Verbatim}

    \hypertarget{integer-division-2-points}{%
\subsection{2. Integer Division (2
Points)}\label{integer-division-2-points}}

What is the difference between dividing using the \texttt{//} operator
and the \texttt{/} operator? For instance, what is the difference
between \texttt{4/2} and \texttt{4//2}?

    \begin{Verbatim}[commandchars=\\\{\}]
{\color{incolor}In [{\color{incolor}26}]:} \PY{c+c1}{\PYZsh{} difference division returns a floating point number by default unless you use the // operator.}
         
         \PY{n+nb}{print}\PY{p}{(}\PY{l+m+mi}{4}\PY{o}{/}\PY{l+m+mi}{2}\PY{p}{)} \PY{c+c1}{\PYZsh{} will return a floating point number}
         \PY{n+nb}{print}\PY{p}{(}\PY{l+m+mi}{4}\PY{o}{/}\PY{o}{/}\PY{l+m+mi}{2}\PY{p}{)} \PY{c+c1}{\PYZsh{} // is floored quotient where it returns an integer }
\end{Verbatim}


    \begin{Verbatim}[commandchars=\\\{\}]
2.0
2

    \end{Verbatim}

    \hypertarget{number-representations-4-points}{%
\subsection{3. Number Representations (4
Points)}\label{number-representations-4-points}}

Pick an integer number between 33 and 126. Print the following
information about this number.

\begin{enumerate}
\def\labelenumi{\arabic{enumi}.}
\tightlist
\item
  Its binary representation
\item
  Its hexadecimal representation
\item
  Its octal representation
\item
  The character corresponding to its Unicode point code.
\end{enumerate}

    \begin{Verbatim}[commandchars=\\\{\}]
{\color{incolor}In [{\color{incolor}34}]:} \PY{c+c1}{\PYZsh{} Its binary representation}
         \PY{n+nb}{int} \PY{o}{=} \PY{l+m+mi}{42}
         
         \PY{n+nb}{print}\PY{p}{(}\PY{n+nb}{bin}\PY{p}{(}\PY{n+nb}{int}\PY{p}{)}\PY{p}{,}\PY{l+s+s2}{\PYZdq{}}\PY{l+s+s2}{in binary.}\PY{l+s+s2}{\PYZdq{}}\PY{p}{)}
\end{Verbatim}


    \begin{Verbatim}[commandchars=\\\{\}]
0b101010 in binary.

    \end{Verbatim}

    \begin{Verbatim}[commandchars=\\\{\}]
{\color{incolor}In [{\color{incolor}35}]:} \PY{c+c1}{\PYZsh{} Its hexadecimal representation}
         \PY{n+nb}{int} \PY{o}{=} \PY{l+m+mi}{42}
         
         \PY{n+nb}{print}\PY{p}{(}\PY{n+nb}{hex}\PY{p}{(}\PY{n+nb}{int}\PY{p}{)}\PY{p}{,}\PY{l+s+s2}{\PYZdq{}}\PY{l+s+s2}{in hexadecimal.}\PY{l+s+s2}{\PYZdq{}}\PY{p}{)}
\end{Verbatim}


    \begin{Verbatim}[commandchars=\\\{\}]
0x2a in hexadecimal.

    \end{Verbatim}

    \begin{Verbatim}[commandchars=\\\{\}]
{\color{incolor}In [{\color{incolor}37}]:} \PY{c+c1}{\PYZsh{} Its octal representation}
         \PY{n+nb}{int} \PY{o}{=} \PY{l+m+mi}{42}
         
         \PY{n+nb}{print}\PY{p}{(}\PY{n+nb}{oct}\PY{p}{(}\PY{n+nb}{int}\PY{p}{)}\PY{p}{,}\PY{l+s+s2}{\PYZdq{}}\PY{l+s+s2}{in octal.}\PY{l+s+s2}{\PYZdq{}}\PY{p}{)}
\end{Verbatim}


    \begin{Verbatim}[commandchars=\\\{\}]
0o52 in octal.

    \end{Verbatim}

    \begin{Verbatim}[commandchars=\\\{\}]
{\color{incolor}In [{\color{incolor}45}]:} \PY{c+c1}{\PYZsh{} The character corresponding to its Unicode point code.}
         
         \PY{n+nb}{print}\PY{p}{(}\PY{l+s+s1}{\PYZsq{}}\PY{l+s+s1}{42}\PY{l+s+s1}{\PYZsq{}}\PY{o}{.}\PY{n}{encode}\PY{p}{(}\PY{l+s+s1}{\PYZsq{}}\PY{l+s+s1}{utf8}\PY{l+s+s1}{\PYZsq{}}\PY{p}{)}\PY{p}{)}  \PY{c+c1}{\PYZsh{} Encoded to 4 bytes in UTF\PYZhy{}8 }
         \PY{n+nb}{print}\PY{p}{(}\PY{l+s+s1}{\PYZsq{}}\PY{l+s+s1}{42}\PY{l+s+s1}{\PYZsq{}}\PY{o}{.}\PY{n}{encode}\PY{p}{(}\PY{l+s+s1}{\PYZsq{}}\PY{l+s+s1}{utf16}\PY{l+s+s1}{\PYZsq{}}\PY{p}{)}\PY{p}{)}  \PY{c+c1}{\PYZsh{} Encoded to 10 bytes in UTF\PYZhy{}16 }
\end{Verbatim}


    \begin{Verbatim}[commandchars=\\\{\}]
b'42'
b'\textbackslash{}xff\textbackslash{}xfe4\textbackslash{}x002\textbackslash{}x00'

    \end{Verbatim}

    \hypertarget{variable-assignment-4-points}{%
\subsection{4. Variable Assignment (4
Points)}\label{variable-assignment-4-points}}

Consider the following two Python code examples. In both cases, we
assign a value to variable \texttt{a}, assign variable \texttt{b} to
\texttt{a} and then make changes variable \texttt{a}. Why is it that in
the first example, changes to \texttt{a} do not affect \texttt{b}, but
in the second example they do?

\emph{Example 1:}

\begin{Shaded}
\begin{Highlighting}[]
\NormalTok{In [}\DecValTok{1}\NormalTok{]:   a }\OperatorTok{=} \DecValTok{1}
\NormalTok{          b }\OperatorTok{=}\NormalTok{ a}
\NormalTok{          a }\OperatorTok{+=} \DecValTok{1}
          \BuiltInTok{print}\NormalTok{(b)}
   
\NormalTok{Out [}\DecValTok{1}\NormalTok{]:  }\DecValTok{1}
\end{Highlighting}
\end{Shaded}

\emph{Example 2:}

\begin{Shaded}
\begin{Highlighting}[]
\NormalTok{In [}\DecValTok{2}\NormalTok{]:   a }\OperatorTok{=}\NormalTok{ [}\DecValTok{1}\NormalTok{, }\DecValTok{2}\NormalTok{, }\DecValTok{3}\NormalTok{]}
\NormalTok{          b }\OperatorTok{=}\NormalTok{ a}
\NormalTok{          a.append(}\DecValTok{4}\NormalTok{)}
          \BuiltInTok{print}\NormalTok{(b)}
   
\NormalTok{Out [}\DecValTok{2}\NormalTok{]:  [}\DecValTok{1}\NormalTok{, }\DecValTok{2}\NormalTok{, }\DecValTok{3}\NormalTok{, }\DecValTok{4}\NormalTok{]}
\end{Highlighting}
\end{Shaded}

    \begin{Verbatim}[commandchars=\\\{\}]
{\color{incolor}In [{\color{incolor}50}]:} \PY{c+c1}{\PYZsh{} Example 1, variables like these are replaced with their values whenever they’re used inside an expression }
         
         \PY{n}{a} \PY{o}{=} \PY{l+m+mi}{4}  \PY{c+c1}{\PYZsh{} We can increment a variable by any number we like.}
         \PY{n}{b} \PY{o}{=} \PY{n}{a}
         \PY{n}{a} \PY{o}{+}\PY{o}{=} \PY{l+m+mi}{1} \PY{c+c1}{\PYZsh{} this shorthand operator += is equivalent to a = a +1}
         \PY{n+nb}{print}\PY{p}{(}\PY{n}{b}\PY{p}{)} 
\end{Verbatim}


    \begin{Verbatim}[commandchars=\\\{\}]
4

    \end{Verbatim}

    \begin{Verbatim}[commandchars=\\\{\}]
{\color{incolor}In [{\color{incolor}47}]:} \PY{c+c1}{\PYZsh{} Example 2, variables are being added to the list versus being sum up}
         
         \PY{n}{a} \PY{o}{=} \PY{p}{[}\PY{l+m+mi}{1}\PY{p}{,} \PY{l+m+mi}{2}\PY{p}{,} \PY{l+m+mi}{3}\PY{p}{]}
         \PY{n}{b} \PY{o}{=} \PY{n}{a}
         \PY{n}{a}\PY{o}{.}\PY{n}{append}\PY{p}{(}\PY{l+m+mi}{4}\PY{p}{)} \PY{c+c1}{\PYZsh{} a new item with value a to the end of the array.}
         \PY{n+nb}{print}\PY{p}{(}\PY{n}{b}\PY{p}{)}
\end{Verbatim}


    \begin{Verbatim}[commandchars=\\\{\}]
[1, 2, 3, 4]

    \end{Verbatim}

    \hypertarget{dynamic-typing-6-points}{%
\subsection{5. Dynamic Typing (6
Points)}\label{dynamic-typing-6-points}}

Static typing vs.~dynamic typing is one of computer programmings most
bitter "\href{http://wiki.c2.com/?HolyWar}{holy wars}'. As a data
scientist, it is important to understand the difference between static
and dynamic typing and the pros/cons of each approach.

Answer each of the following questions in your own words.

\begin{enumerate}
\def\labelenumi{\alph{enumi}.}
\item
  What is the difference between static and dynamic typing?
\item
  What are the benefits of static typing over dynamic typing?
\item
  What are the benefits of dynamic typing over static typing?
\end{enumerate}

    Static typing is when your type checking occurs at compile time. You
must define a type for your variables inside of your code and any
operations you perform on your data would be checked by the compiler.

Dynamic typing is when your type checking occurs at runtime. Instead of
errors coming up when you compile your code you will get runtime errors
if you try performing operations on incompatible types. However, you
will get the benefit of having more versatile functions as they can be
written once for multiple data types. Benefits over Static: More
succinct/less verbose and you spend less time debugging syntax and
semantic errors.

Static typing catches errors early, instead of finding them during
execution (especially useful for long programs). It's more ``strict'' in
that it won't allow for type errors anywhere in your program and often
prevents variables from changing types, which further defends against
unintended errors. Dynamic typing is more flexible (which some
appreciate) but allows for variables to change types (sometimes creating
unexpected errors). Draw back of static: Like any formalism, types
require some investment up front to become fluent in. Some tasks,
especially around generic programming, can be very easily expressed in a
dynamic language, but require more machinery in a static language.

Static pros: can ease the mental burden of writing programs, by
automatically tracking information the programmer would otherwise have
to track mentally in some fashion. More errors detected earlier in
development, fewer errors at runtime and in shipped code and no need to
write entirely mechanical tests for type correctness.

DYNAMIC PROS: Deals naturally with certain types of self-describing
data, code can be use polymorphically without programmer decoration and
tends to reduce unnecessary clutter and duplication/repetition in code.

    \hypertarget{garbage-collection-4-points}{%
\subsection{6. Garbage Collection (4
Points)}\label{garbage-collection-4-points}}

\begin{enumerate}
\def\labelenumi{\alph{enumi}.}
\item
  Explain what \emph{garbage collection} means in connection to
  programming languages.
\item
  How does CPython implement garbage collection?
\end{enumerate}

    Python garbage collection feature cleans up unused memory as your
program runs and frees you from having to manage such details in your
code. The space is reclaimed immediately, as soon as the last reference
to an object is removed. Unfortunately, classical reference counting has
a fundamental problem --- it cannot detect reference cycles.

CPython uses a threshold based garbage collector. This means actual gc
will happen only when some object thresholds are met. An explicit call
to gc.collect will release the memory. Standard CPython's garbage
collector has two components, the reference counting collector and the
generational garbage collector, known as gc module. The reference
counting algorithm is incredibly efficient and straightforward, but it
cannot detect reference cycles. That is why Python has a supplemental
algorithm called generational cyclic GC, that deals with reference
cycles. The reference counting is fundamental to Python and can't be
disabled, whereas the cyclic GC is optional and can be used manually.


    % Add a bibliography block to the postdoc
    
    
    
    \end{document}
