
% Default to the notebook output style

    


% Inherit from the specified cell style.




    
\documentclass[11pt]{article}

    
    
    \usepackage[T1]{fontenc}
    % Nicer default font (+ math font) than Computer Modern for most use cases
    \usepackage{mathpazo}

    % Basic figure setup, for now with no caption control since it's done
    % automatically by Pandoc (which extracts ![](path) syntax from Markdown).
    \usepackage{graphicx}
    % We will generate all images so they have a width \maxwidth. This means
    % that they will get their normal width if they fit onto the page, but
    % are scaled down if they would overflow the margins.
    \makeatletter
    \def\maxwidth{\ifdim\Gin@nat@width>\linewidth\linewidth
    \else\Gin@nat@width\fi}
    \makeatother
    \let\Oldincludegraphics\includegraphics
    % Set max figure width to be 80% of text width, for now hardcoded.
    \renewcommand{\includegraphics}[1]{\Oldincludegraphics[width=.8\maxwidth]{#1}}
    % Ensure that by default, figures have no caption (until we provide a
    % proper Figure object with a Caption API and a way to capture that
    % in the conversion process - todo).
    \usepackage{caption}
    \DeclareCaptionLabelFormat{nolabel}{}
    \captionsetup{labelformat=nolabel}

    \usepackage{adjustbox} % Used to constrain images to a maximum size 
    \usepackage{xcolor} % Allow colors to be defined
    \usepackage{enumerate} % Needed for markdown enumerations to work
    \usepackage{geometry} % Used to adjust the document margins
    \usepackage{amsmath} % Equations
    \usepackage{amssymb} % Equations
    \usepackage{textcomp} % defines textquotesingle
    % Hack from http://tex.stackexchange.com/a/47451/13684:
    \AtBeginDocument{%
        \def\PYZsq{\textquotesingle}% Upright quotes in Pygmentized code
    }
    \usepackage{upquote} % Upright quotes for verbatim code
    \usepackage{eurosym} % defines \euro
    \usepackage[mathletters]{ucs} % Extended unicode (utf-8) support
    \usepackage[utf8x]{inputenc} % Allow utf-8 characters in the tex document
    \usepackage{fancyvrb} % verbatim replacement that allows latex
    \usepackage{grffile} % extends the file name processing of package graphics 
                         % to support a larger range 
    % The hyperref package gives us a pdf with properly built
    % internal navigation ('pdf bookmarks' for the table of contents,
    % internal cross-reference links, web links for URLs, etc.)
    \usepackage{hyperref}
    \usepackage{longtable} % longtable support required by pandoc >1.10
    \usepackage{booktabs}  % table support for pandoc > 1.12.2
    \usepackage[inline]{enumitem} % IRkernel/repr support (it uses the enumerate* environment)
    \usepackage[normalem]{ulem} % ulem is needed to support strikethroughs (\sout)
                                % normalem makes italics be italics, not underlines
    

    
    
    % Colors for the hyperref package
    \definecolor{urlcolor}{rgb}{0,.145,.698}
    \definecolor{linkcolor}{rgb}{.71,0.21,0.01}
    \definecolor{citecolor}{rgb}{.12,.54,.11}

    % ANSI colors
    \definecolor{ansi-black}{HTML}{3E424D}
    \definecolor{ansi-black-intense}{HTML}{282C36}
    \definecolor{ansi-red}{HTML}{E75C58}
    \definecolor{ansi-red-intense}{HTML}{B22B31}
    \definecolor{ansi-green}{HTML}{00A250}
    \definecolor{ansi-green-intense}{HTML}{007427}
    \definecolor{ansi-yellow}{HTML}{DDB62B}
    \definecolor{ansi-yellow-intense}{HTML}{B27D12}
    \definecolor{ansi-blue}{HTML}{208FFB}
    \definecolor{ansi-blue-intense}{HTML}{0065CA}
    \definecolor{ansi-magenta}{HTML}{D160C4}
    \definecolor{ansi-magenta-intense}{HTML}{A03196}
    \definecolor{ansi-cyan}{HTML}{60C6C8}
    \definecolor{ansi-cyan-intense}{HTML}{258F8F}
    \definecolor{ansi-white}{HTML}{C5C1B4}
    \definecolor{ansi-white-intense}{HTML}{A1A6B2}

    % commands and environments needed by pandoc snippets
    % extracted from the output of `pandoc -s`
    \providecommand{\tightlist}{%
      \setlength{\itemsep}{0pt}\setlength{\parskip}{0pt}}
    \DefineVerbatimEnvironment{Highlighting}{Verbatim}{commandchars=\\\{\}}
    % Add ',fontsize=\small' for more characters per line
    \newenvironment{Shaded}{}{}
    \newcommand{\KeywordTok}[1]{\textcolor[rgb]{0.00,0.44,0.13}{\textbf{{#1}}}}
    \newcommand{\DataTypeTok}[1]{\textcolor[rgb]{0.56,0.13,0.00}{{#1}}}
    \newcommand{\DecValTok}[1]{\textcolor[rgb]{0.25,0.63,0.44}{{#1}}}
    \newcommand{\BaseNTok}[1]{\textcolor[rgb]{0.25,0.63,0.44}{{#1}}}
    \newcommand{\FloatTok}[1]{\textcolor[rgb]{0.25,0.63,0.44}{{#1}}}
    \newcommand{\CharTok}[1]{\textcolor[rgb]{0.25,0.44,0.63}{{#1}}}
    \newcommand{\StringTok}[1]{\textcolor[rgb]{0.25,0.44,0.63}{{#1}}}
    \newcommand{\CommentTok}[1]{\textcolor[rgb]{0.38,0.63,0.69}{\textit{{#1}}}}
    \newcommand{\OtherTok}[1]{\textcolor[rgb]{0.00,0.44,0.13}{{#1}}}
    \newcommand{\AlertTok}[1]{\textcolor[rgb]{1.00,0.00,0.00}{\textbf{{#1}}}}
    \newcommand{\FunctionTok}[1]{\textcolor[rgb]{0.02,0.16,0.49}{{#1}}}
    \newcommand{\RegionMarkerTok}[1]{{#1}}
    \newcommand{\ErrorTok}[1]{\textcolor[rgb]{1.00,0.00,0.00}{\textbf{{#1}}}}
    \newcommand{\NormalTok}[1]{{#1}}
    
    % Additional commands for more recent versions of Pandoc
    \newcommand{\ConstantTok}[1]{\textcolor[rgb]{0.53,0.00,0.00}{{#1}}}
    \newcommand{\SpecialCharTok}[1]{\textcolor[rgb]{0.25,0.44,0.63}{{#1}}}
    \newcommand{\VerbatimStringTok}[1]{\textcolor[rgb]{0.25,0.44,0.63}{{#1}}}
    \newcommand{\SpecialStringTok}[1]{\textcolor[rgb]{0.73,0.40,0.53}{{#1}}}
    \newcommand{\ImportTok}[1]{{#1}}
    \newcommand{\DocumentationTok}[1]{\textcolor[rgb]{0.73,0.13,0.13}{\textit{{#1}}}}
    \newcommand{\AnnotationTok}[1]{\textcolor[rgb]{0.38,0.63,0.69}{\textbf{\textit{{#1}}}}}
    \newcommand{\CommentVarTok}[1]{\textcolor[rgb]{0.38,0.63,0.69}{\textbf{\textit{{#1}}}}}
    \newcommand{\VariableTok}[1]{\textcolor[rgb]{0.10,0.09,0.49}{{#1}}}
    \newcommand{\ControlFlowTok}[1]{\textcolor[rgb]{0.00,0.44,0.13}{\textbf{{#1}}}}
    \newcommand{\OperatorTok}[1]{\textcolor[rgb]{0.40,0.40,0.40}{{#1}}}
    \newcommand{\BuiltInTok}[1]{{#1}}
    \newcommand{\ExtensionTok}[1]{{#1}}
    \newcommand{\PreprocessorTok}[1]{\textcolor[rgb]{0.74,0.48,0.00}{{#1}}}
    \newcommand{\AttributeTok}[1]{\textcolor[rgb]{0.49,0.56,0.16}{{#1}}}
    \newcommand{\InformationTok}[1]{\textcolor[rgb]{0.38,0.63,0.69}{\textbf{\textit{{#1}}}}}
    \newcommand{\WarningTok}[1]{\textcolor[rgb]{0.38,0.63,0.69}{\textbf{\textit{{#1}}}}}
    
    
    % Define a nice break command that doesn't care if a line doesn't already
    % exist.
    \def\br{\hspace*{\fill} \\* }
    % Math Jax compatability definitions
    \def\gt{>}
    \def\lt{<}
    % Document parameters
    \title{DSC-510 Assignment 3 CHERRERA}
    
    
    

    % Pygments definitions
    
\makeatletter
\def\PY@reset{\let\PY@it=\relax \let\PY@bf=\relax%
    \let\PY@ul=\relax \let\PY@tc=\relax%
    \let\PY@bc=\relax \let\PY@ff=\relax}
\def\PY@tok#1{\csname PY@tok@#1\endcsname}
\def\PY@toks#1+{\ifx\relax#1\empty\else%
    \PY@tok{#1}\expandafter\PY@toks\fi}
\def\PY@do#1{\PY@bc{\PY@tc{\PY@ul{%
    \PY@it{\PY@bf{\PY@ff{#1}}}}}}}
\def\PY#1#2{\PY@reset\PY@toks#1+\relax+\PY@do{#2}}

\expandafter\def\csname PY@tok@w\endcsname{\def\PY@tc##1{\textcolor[rgb]{0.73,0.73,0.73}{##1}}}
\expandafter\def\csname PY@tok@c\endcsname{\let\PY@it=\textit\def\PY@tc##1{\textcolor[rgb]{0.25,0.50,0.50}{##1}}}
\expandafter\def\csname PY@tok@cp\endcsname{\def\PY@tc##1{\textcolor[rgb]{0.74,0.48,0.00}{##1}}}
\expandafter\def\csname PY@tok@k\endcsname{\let\PY@bf=\textbf\def\PY@tc##1{\textcolor[rgb]{0.00,0.50,0.00}{##1}}}
\expandafter\def\csname PY@tok@kp\endcsname{\def\PY@tc##1{\textcolor[rgb]{0.00,0.50,0.00}{##1}}}
\expandafter\def\csname PY@tok@kt\endcsname{\def\PY@tc##1{\textcolor[rgb]{0.69,0.00,0.25}{##1}}}
\expandafter\def\csname PY@tok@o\endcsname{\def\PY@tc##1{\textcolor[rgb]{0.40,0.40,0.40}{##1}}}
\expandafter\def\csname PY@tok@ow\endcsname{\let\PY@bf=\textbf\def\PY@tc##1{\textcolor[rgb]{0.67,0.13,1.00}{##1}}}
\expandafter\def\csname PY@tok@nb\endcsname{\def\PY@tc##1{\textcolor[rgb]{0.00,0.50,0.00}{##1}}}
\expandafter\def\csname PY@tok@nf\endcsname{\def\PY@tc##1{\textcolor[rgb]{0.00,0.00,1.00}{##1}}}
\expandafter\def\csname PY@tok@nc\endcsname{\let\PY@bf=\textbf\def\PY@tc##1{\textcolor[rgb]{0.00,0.00,1.00}{##1}}}
\expandafter\def\csname PY@tok@nn\endcsname{\let\PY@bf=\textbf\def\PY@tc##1{\textcolor[rgb]{0.00,0.00,1.00}{##1}}}
\expandafter\def\csname PY@tok@ne\endcsname{\let\PY@bf=\textbf\def\PY@tc##1{\textcolor[rgb]{0.82,0.25,0.23}{##1}}}
\expandafter\def\csname PY@tok@nv\endcsname{\def\PY@tc##1{\textcolor[rgb]{0.10,0.09,0.49}{##1}}}
\expandafter\def\csname PY@tok@no\endcsname{\def\PY@tc##1{\textcolor[rgb]{0.53,0.00,0.00}{##1}}}
\expandafter\def\csname PY@tok@nl\endcsname{\def\PY@tc##1{\textcolor[rgb]{0.63,0.63,0.00}{##1}}}
\expandafter\def\csname PY@tok@ni\endcsname{\let\PY@bf=\textbf\def\PY@tc##1{\textcolor[rgb]{0.60,0.60,0.60}{##1}}}
\expandafter\def\csname PY@tok@na\endcsname{\def\PY@tc##1{\textcolor[rgb]{0.49,0.56,0.16}{##1}}}
\expandafter\def\csname PY@tok@nt\endcsname{\let\PY@bf=\textbf\def\PY@tc##1{\textcolor[rgb]{0.00,0.50,0.00}{##1}}}
\expandafter\def\csname PY@tok@nd\endcsname{\def\PY@tc##1{\textcolor[rgb]{0.67,0.13,1.00}{##1}}}
\expandafter\def\csname PY@tok@s\endcsname{\def\PY@tc##1{\textcolor[rgb]{0.73,0.13,0.13}{##1}}}
\expandafter\def\csname PY@tok@sd\endcsname{\let\PY@it=\textit\def\PY@tc##1{\textcolor[rgb]{0.73,0.13,0.13}{##1}}}
\expandafter\def\csname PY@tok@si\endcsname{\let\PY@bf=\textbf\def\PY@tc##1{\textcolor[rgb]{0.73,0.40,0.53}{##1}}}
\expandafter\def\csname PY@tok@se\endcsname{\let\PY@bf=\textbf\def\PY@tc##1{\textcolor[rgb]{0.73,0.40,0.13}{##1}}}
\expandafter\def\csname PY@tok@sr\endcsname{\def\PY@tc##1{\textcolor[rgb]{0.73,0.40,0.53}{##1}}}
\expandafter\def\csname PY@tok@ss\endcsname{\def\PY@tc##1{\textcolor[rgb]{0.10,0.09,0.49}{##1}}}
\expandafter\def\csname PY@tok@sx\endcsname{\def\PY@tc##1{\textcolor[rgb]{0.00,0.50,0.00}{##1}}}
\expandafter\def\csname PY@tok@m\endcsname{\def\PY@tc##1{\textcolor[rgb]{0.40,0.40,0.40}{##1}}}
\expandafter\def\csname PY@tok@gh\endcsname{\let\PY@bf=\textbf\def\PY@tc##1{\textcolor[rgb]{0.00,0.00,0.50}{##1}}}
\expandafter\def\csname PY@tok@gu\endcsname{\let\PY@bf=\textbf\def\PY@tc##1{\textcolor[rgb]{0.50,0.00,0.50}{##1}}}
\expandafter\def\csname PY@tok@gd\endcsname{\def\PY@tc##1{\textcolor[rgb]{0.63,0.00,0.00}{##1}}}
\expandafter\def\csname PY@tok@gi\endcsname{\def\PY@tc##1{\textcolor[rgb]{0.00,0.63,0.00}{##1}}}
\expandafter\def\csname PY@tok@gr\endcsname{\def\PY@tc##1{\textcolor[rgb]{1.00,0.00,0.00}{##1}}}
\expandafter\def\csname PY@tok@ge\endcsname{\let\PY@it=\textit}
\expandafter\def\csname PY@tok@gs\endcsname{\let\PY@bf=\textbf}
\expandafter\def\csname PY@tok@gp\endcsname{\let\PY@bf=\textbf\def\PY@tc##1{\textcolor[rgb]{0.00,0.00,0.50}{##1}}}
\expandafter\def\csname PY@tok@go\endcsname{\def\PY@tc##1{\textcolor[rgb]{0.53,0.53,0.53}{##1}}}
\expandafter\def\csname PY@tok@gt\endcsname{\def\PY@tc##1{\textcolor[rgb]{0.00,0.27,0.87}{##1}}}
\expandafter\def\csname PY@tok@err\endcsname{\def\PY@bc##1{\setlength{\fboxsep}{0pt}\fcolorbox[rgb]{1.00,0.00,0.00}{1,1,1}{\strut ##1}}}
\expandafter\def\csname PY@tok@kc\endcsname{\let\PY@bf=\textbf\def\PY@tc##1{\textcolor[rgb]{0.00,0.50,0.00}{##1}}}
\expandafter\def\csname PY@tok@kd\endcsname{\let\PY@bf=\textbf\def\PY@tc##1{\textcolor[rgb]{0.00,0.50,0.00}{##1}}}
\expandafter\def\csname PY@tok@kn\endcsname{\let\PY@bf=\textbf\def\PY@tc##1{\textcolor[rgb]{0.00,0.50,0.00}{##1}}}
\expandafter\def\csname PY@tok@kr\endcsname{\let\PY@bf=\textbf\def\PY@tc##1{\textcolor[rgb]{0.00,0.50,0.00}{##1}}}
\expandafter\def\csname PY@tok@bp\endcsname{\def\PY@tc##1{\textcolor[rgb]{0.00,0.50,0.00}{##1}}}
\expandafter\def\csname PY@tok@fm\endcsname{\def\PY@tc##1{\textcolor[rgb]{0.00,0.00,1.00}{##1}}}
\expandafter\def\csname PY@tok@vc\endcsname{\def\PY@tc##1{\textcolor[rgb]{0.10,0.09,0.49}{##1}}}
\expandafter\def\csname PY@tok@vg\endcsname{\def\PY@tc##1{\textcolor[rgb]{0.10,0.09,0.49}{##1}}}
\expandafter\def\csname PY@tok@vi\endcsname{\def\PY@tc##1{\textcolor[rgb]{0.10,0.09,0.49}{##1}}}
\expandafter\def\csname PY@tok@vm\endcsname{\def\PY@tc##1{\textcolor[rgb]{0.10,0.09,0.49}{##1}}}
\expandafter\def\csname PY@tok@sa\endcsname{\def\PY@tc##1{\textcolor[rgb]{0.73,0.13,0.13}{##1}}}
\expandafter\def\csname PY@tok@sb\endcsname{\def\PY@tc##1{\textcolor[rgb]{0.73,0.13,0.13}{##1}}}
\expandafter\def\csname PY@tok@sc\endcsname{\def\PY@tc##1{\textcolor[rgb]{0.73,0.13,0.13}{##1}}}
\expandafter\def\csname PY@tok@dl\endcsname{\def\PY@tc##1{\textcolor[rgb]{0.73,0.13,0.13}{##1}}}
\expandafter\def\csname PY@tok@s2\endcsname{\def\PY@tc##1{\textcolor[rgb]{0.73,0.13,0.13}{##1}}}
\expandafter\def\csname PY@tok@sh\endcsname{\def\PY@tc##1{\textcolor[rgb]{0.73,0.13,0.13}{##1}}}
\expandafter\def\csname PY@tok@s1\endcsname{\def\PY@tc##1{\textcolor[rgb]{0.73,0.13,0.13}{##1}}}
\expandafter\def\csname PY@tok@mb\endcsname{\def\PY@tc##1{\textcolor[rgb]{0.40,0.40,0.40}{##1}}}
\expandafter\def\csname PY@tok@mf\endcsname{\def\PY@tc##1{\textcolor[rgb]{0.40,0.40,0.40}{##1}}}
\expandafter\def\csname PY@tok@mh\endcsname{\def\PY@tc##1{\textcolor[rgb]{0.40,0.40,0.40}{##1}}}
\expandafter\def\csname PY@tok@mi\endcsname{\def\PY@tc##1{\textcolor[rgb]{0.40,0.40,0.40}{##1}}}
\expandafter\def\csname PY@tok@il\endcsname{\def\PY@tc##1{\textcolor[rgb]{0.40,0.40,0.40}{##1}}}
\expandafter\def\csname PY@tok@mo\endcsname{\def\PY@tc##1{\textcolor[rgb]{0.40,0.40,0.40}{##1}}}
\expandafter\def\csname PY@tok@ch\endcsname{\let\PY@it=\textit\def\PY@tc##1{\textcolor[rgb]{0.25,0.50,0.50}{##1}}}
\expandafter\def\csname PY@tok@cm\endcsname{\let\PY@it=\textit\def\PY@tc##1{\textcolor[rgb]{0.25,0.50,0.50}{##1}}}
\expandafter\def\csname PY@tok@cpf\endcsname{\let\PY@it=\textit\def\PY@tc##1{\textcolor[rgb]{0.25,0.50,0.50}{##1}}}
\expandafter\def\csname PY@tok@c1\endcsname{\let\PY@it=\textit\def\PY@tc##1{\textcolor[rgb]{0.25,0.50,0.50}{##1}}}
\expandafter\def\csname PY@tok@cs\endcsname{\let\PY@it=\textit\def\PY@tc##1{\textcolor[rgb]{0.25,0.50,0.50}{##1}}}

\def\PYZbs{\char`\\}
\def\PYZus{\char`\_}
\def\PYZob{\char`\{}
\def\PYZcb{\char`\}}
\def\PYZca{\char`\^}
\def\PYZam{\char`\&}
\def\PYZlt{\char`\<}
\def\PYZgt{\char`\>}
\def\PYZsh{\char`\#}
\def\PYZpc{\char`\%}
\def\PYZdl{\char`\$}
\def\PYZhy{\char`\-}
\def\PYZsq{\char`\'}
\def\PYZdq{\char`\"}
\def\PYZti{\char`\~}
% for compatibility with earlier versions
\def\PYZat{@}
\def\PYZlb{[}
\def\PYZrb{]}
\makeatother


    % Exact colors from NB
    \definecolor{incolor}{rgb}{0.0, 0.0, 0.5}
    \definecolor{outcolor}{rgb}{0.545, 0.0, 0.0}



    
    % Prevent overflowing lines due to hard-to-break entities
    \sloppy 
    % Setup hyperref package
    \hypersetup{
      breaklinks=true,  % so long urls are correctly broken across lines
      colorlinks=true,
      urlcolor=urlcolor,
      linkcolor=linkcolor,
      citecolor=citecolor,
      }
    % Slightly bigger margins than the latex defaults
    
    \geometry{verbose,tmargin=1in,bmargin=1in,lmargin=1in,rmargin=1in}
    
    

    \begin{document}
    
    
    \maketitle
    
    

    
    \hypertarget{assignment-3---strings-and-collection-types}{%
\section{Assignment 3 - Strings and Collection
Types}\label{assignment-3---strings-and-collection-types}}

\hypertarget{instructions}{%
\subsection{Instructions}\label{instructions}}

For this assignment we will be working with the \texttt{avengers.csv}
dataset used for the
\href{fivethirtyeight.com/features/avengers-death-comics-age-of-ultron}{Joining
The Avengers Is As Deadly As Jumping Off A Four-Story Building}. You can
find the original data in the
\href{https://github.com/fivethirtyeight/data/tree/master/avengers}{fivethirtyeight/data}
Github repository. Copy this data into the same directory as your
assignment notebook.

Follow the instructions for submitting a Jupyter Notebook assignment in
the submitting assignments documentation.

    \hypertarget{reading-data-from-a-file-5-points}{%
\subsection{1. Reading data from a file (5
points)}\label{reading-data-from-a-file-5-points}}

The \texttt{avengers.csv} is encoded with ISO-8859-1 character encoding.
This is common in CSV files generated by Microsoft Excel.

Try opening the the file \texttt{avengers.csv} and reading the data.
What error message do you receive? What does it mean?

    \begin{Verbatim}[commandchars=\\\{\}]
{\color{incolor}In [{\color{incolor}10}]:} \PY{k}{with} \PY{n+nb}{open}\PY{p}{(}\PY{l+s+s1}{\PYZsq{}}\PY{l+s+s1}{avengers.csv}\PY{l+s+s1}{\PYZsq{}}\PY{p}{,} \PY{l+s+s1}{\PYZsq{}}\PY{l+s+s1}{r}\PY{l+s+s1}{\PYZsq{}}\PY{p}{)} \PY{k}{as} \PY{n}{f}\PY{p}{:} \PY{c+c1}{\PYZsh{} I had to do some additional adjustments with saving file as comma seperated CSV file}
             \PY{n}{lines} \PY{o}{=} \PY{n}{f}\PY{o}{.}\PY{n}{readlines}\PY{p}{(}\PY{p}{)}   
\end{Verbatim}


    \hypertarget{changing-file-encodings-5-points}{%
\subsection{2. Changing file encodings (5
points)}\label{changing-file-encodings-5-points}}

In order to work with the data in \texttt{avengers.csv}, we need to
change its character encoding to \texttt{utf-8}.

Open the \texttt{avengers.csv} as a binary file, decode the binary data
from \texttt{ISO-8859-1}, and write the \texttt{utf-8} encoded data to
\texttt{avengers\_utf.csv}.

    \begin{Verbatim}[commandchars=\\\{\}]
{\color{incolor}In [{\color{incolor}16}]:} \PY{k}{with} \PY{n+nb}{open}\PY{p}{(}\PY{l+s+s1}{\PYZsq{}}\PY{l+s+s1}{avengers.csv}\PY{l+s+s1}{\PYZsq{}}\PY{p}{,} \PY{l+s+s1}{\PYZsq{}}\PY{l+s+s1}{rb}\PY{l+s+s1}{\PYZsq{}}\PY{p}{)} \PY{k}{as} \PY{n}{f}\PY{p}{:} 
             \PY{n}{data} \PY{o}{=} \PY{n}{f}\PY{o}{.}\PY{n}{read}\PY{p}{(}\PY{p}{)}
             
         \PY{n}{decoded\PYZus{}data} \PY{o}{=} \PY{n}{data}\PY{o}{.}\PY{n}{decode}\PY{p}{(}\PY{l+s+s1}{\PYZsq{}}\PY{l+s+s1}{iso\PYZhy{}8859\PYZhy{}1}\PY{l+s+s1}{\PYZsq{}}\PY{p}{)}
         
         \PY{k}{with} \PY{n+nb}{open}\PY{p}{(}\PY{l+s+s1}{\PYZsq{}}\PY{l+s+s1}{avengers\PYZus{}utf8.csv}\PY{l+s+s1}{\PYZsq{}}\PY{p}{,} \PY{l+s+s1}{\PYZsq{}}\PY{l+s+s1}{w}\PY{l+s+s1}{\PYZsq{}}\PY{p}{)} \PY{k}{as} \PY{n}{f}\PY{p}{:}
             \PY{n}{f}\PY{o}{.}\PY{n}{write}\PY{p}{(}\PY{n}{decoded\PYZus{}data}\PY{p}{)}
\end{Verbatim}


    \hypertarget{read-and-count-the-lines-3-points}{%
\subsection{3. Read and count the lines (3
points)}\label{read-and-count-the-lines-3-points}}

Open the \texttt{resouces/avengers\_utf8.csv} data and read the lines
from the file. Assign the lines to the variable \texttt{lines}.

How many lines does the file contain?

    \begin{Verbatim}[commandchars=\\\{\}]
{\color{incolor}In [{\color{incolor}17}]:} \PY{k}{with} \PY{n+nb}{open}\PY{p}{(}\PY{l+s+s1}{\PYZsq{}}\PY{l+s+s1}{avengers\PYZus{}utf8.csv}\PY{l+s+s1}{\PYZsq{}}\PY{p}{)} \PY{k}{as} \PY{n}{f}\PY{p}{:}
                   \PY{n}{avengers} \PY{o}{=} \PY{n}{f}\PY{o}{.}\PY{n}{readlines}\PY{p}{(}\PY{p}{)}
                   
         \PY{n+nb}{print}\PY{p}{(}\PY{n+nb}{len}\PY{p}{(}\PY{n}{avengers}\PY{p}{)}\PY{p}{)} \PY{c+c1}{\PYZsh{}It returns back 174 lines}
\end{Verbatim}


    \begin{Verbatim}[commandchars=\\\{\}]
174

    \end{Verbatim}

    \hypertarget{parse-the-header-row-12-points}{%
\subsection{4. Parse the header row (12
Points)}\label{parse-the-header-row-12-points}}

The first row of the CSV file is the header row.

\begin{enumerate}
\def\labelenumi{\arabic{enumi}.}
\tightlist
\item
  Using list slicing, assign the header row to the variable
  \texttt{header\_row}.\\
\item
  CSV files use commas to separate fields. Create a list of header
  fields from the \texttt{header\_row} variable using the string
  \texttt{split} method.\\
\item
  Using list slicing, make a list that contains last two fields in the
  header.
\item
  Using list slicing, make a list that contains the 3rd through 6th
  elements (Hint: Remember that Python uses zero-based indexing).
\end{enumerate}

    \begin{Verbatim}[commandchars=\\\{\}]
{\color{incolor}In [{\color{incolor}29}]:} \PY{n}{header\PYZus{}row} \PY{o}{=} \PY{n}{avengers}\PY{p}{[}\PY{l+m+mi}{0}\PY{p}{]}\PY{o}{.}\PY{n}{split}\PY{p}{(}\PY{l+s+s1}{\PYZsq{}}\PY{l+s+s1}{,}\PY{l+s+s1}{\PYZsq{}}\PY{p}{)} \PY{c+c1}{\PYZsh{} creating a list by seperating by the comma}
         \PY{n+nb}{print}\PY{p}{(}\PY{n}{header\PYZus{}row}\PY{p}{)}
         \PY{n}{last\PYZus{}two} \PY{o}{=} \PY{n}{header\PYZus{}row}\PY{p}{[}\PY{l+m+mi}{19}\PY{p}{:}\PY{l+m+mi}{22}\PY{p}{]} \PY{c+c1}{\PYZsh{} making list with 19 is the starting and 22 is at ending point but doesn\PYZsq{}t include it, if 1 represent grabing each item}
         \PY{n+nb}{print}\PY{p}{(}\PY{n}{last\PYZus{}two}\PY{p}{)}
         \PY{n}{three\PYZus{}through\PYZus{}six} \PY{o}{=} \PY{n}{header\PYZus{}row}\PY{p}{[}\PY{l+m+mi}{2}\PY{p}{:}\PY{l+m+mi}{6}\PY{p}{]} \PY{c+c1}{\PYZsh{} making list 3\PYZhy{}6, remember first number is counted as zero}
         \PY{n+nb}{print}\PY{p}{(}\PY{n}{three\PYZus{}through\PYZus{}six}\PY{p}{)}
\end{Verbatim}


    \begin{Verbatim}[commandchars=\\\{\}]
['URL', 'Name/Alias', 'Appearances', 'Current?', 'Gender', 'Probationary Introl', 'Full/Reserve Avengers Intro', 'Year', 'Years since joining', 'Honorary', 'Death1', 'Return1', 'Death2', 'Return2', 'Death3', 'Return3', 'Death4', 'Return4', 'Death5', 'Return5', 'Notes\textbackslash{}n']
['Return5', 'Notes\textbackslash{}n']
['Appearances', 'Current?', 'Gender', 'Probationary Introl']

    \end{Verbatim}

    \hypertarget{working-with-tuples-6-points}{%
\subsection{5. Working with Tuples (6
Points)}\label{working-with-tuples-6-points}}

Using the \texttt{header} variable created in the previous problem, we
are going to create a tuple from that header and assign it to the
variable \texttt{header\_tuple}. We also create a copy of the original
header and assign it to \texttt{header\_copy}.

\begin{Shaded}
\begin{Highlighting}[]
\ImportTok{import}\NormalTok{ copy }

\NormalTok{header_tuple }\OperatorTok{=} \BuiltInTok{tuple}\NormalTok{(header)}
\NormalTok{header_copy }\OperatorTok{=}\NormalTok{ copy.copy(header)}
\end{Highlighting}
\end{Shaded}

\begin{enumerate}
\def\labelenumi{\arabic{enumi}.}
\tightlist
\item
  In the original header, the last value in the list has an extra
  newline. Strip the newline character from the last header field and
  reassign..
\item
  In the \texttt{header\_copy} list, append the value
  \texttt{More\ info} to the end of the list. Verify the value has been
  added.
\item
  Change the value of the first item in the header list from
  \texttt{URL} to \texttt{url}. Verify the value has been changed.
\item
  Try steps 1 and 2 on \texttt{header\_tuple} instead of
  \texttt{header}. What happens? Why can we change \texttt{header}, but
  not \texttt{header\_tuple}?
\end{enumerate}

    \begin{Verbatim}[commandchars=\\\{\}]
{\color{incolor}In [{\color{incolor}31}]:} \PY{n}{header\PYZus{}row} \PY{o}{=} \PY{n}{avengers}\PY{p}{[}\PY{l+m+mi}{0}\PY{p}{]}\PY{o}{.}\PY{n}{strip}\PY{p}{(}\PY{l+s+s1}{\PYZsq{}}\PY{l+s+se}{\PYZbs{}n}\PY{l+s+s1}{\PYZsq{}}\PY{p}{)}\PY{o}{.}\PY{n}{split}\PY{p}{(}\PY{l+s+s1}{\PYZsq{}}\PY{l+s+s1}{,}\PY{l+s+s1}{\PYZsq{}}\PY{p}{)} \PY{c+c1}{\PYZsh{} strip the newline character}
         \PY{n+nb}{print}\PY{p}{(}\PY{n}{header\PYZus{}row}\PY{p}{)}
         \PY{n}{last\PYZus{}two} \PY{o}{=} \PY{n}{header\PYZus{}row}\PY{p}{[}\PY{l+m+mi}{19}\PY{p}{:}\PY{l+m+mi}{22}\PY{p}{]} 
         \PY{n+nb}{print}\PY{p}{(}\PY{n}{last\PYZus{}two}\PY{p}{)}
         \PY{n}{three\PYZus{}through\PYZus{}six} \PY{o}{=} \PY{n}{header\PYZus{}row}\PY{p}{[}\PY{l+m+mi}{2}\PY{p}{:}\PY{l+m+mi}{6}\PY{p}{]} 
         \PY{n+nb}{print}\PY{p}{(}\PY{n}{three\PYZus{}through\PYZus{}six}\PY{p}{)} 
\end{Verbatim}


    \begin{Verbatim}[commandchars=\\\{\}]
['URL', 'Name/Alias', 'Appearances', 'Current?', 'Gender', 'Probationary Introl', 'Full/Reserve Avengers Intro', 'Year', 'Years since joining', 'Honorary', 'Death1', 'Return1', 'Death2', 'Return2', 'Death3', 'Return3', 'Death4', 'Return4', 'Death5', 'Return5', 'Notes']
['Return5', 'Notes']
['Appearances', 'Current?', 'Gender', 'Probationary Introl']

    \end{Verbatim}

    \begin{Verbatim}[commandchars=\\\{\}]
{\color{incolor}In [{\color{incolor}32}]:} \PY{n+nb}{print}\PY{p}{(}\PY{n}{header\PYZus{}row}\PY{p}{)}
\end{Verbatim}


    \begin{Verbatim}[commandchars=\\\{\}]
['URL', 'Name/Alias', 'Appearances', 'Current?', 'Gender', 'Probationary Introl', 'Full/Reserve Avengers Intro', 'Year', 'Years since joining', 'Honorary', 'Death1', 'Return1', 'Death2', 'Return2', 'Death3', 'Return3', 'Death4', 'Return4', 'Death5', 'Return5', 'Notes']

    \end{Verbatim}

    \begin{Verbatim}[commandchars=\\\{\}]
{\color{incolor}In [{\color{incolor}39}]:} \PY{n+nb}{print}\PY{p}{(}\PY{n}{header\PYZus{}row}\PY{p}{)}
         
         \PY{n}{header\PYZus{}row}\PY{o}{.}\PY{n}{append}\PY{p}{(}\PY{l+s+s1}{\PYZsq{}}\PY{l+s+s1}{More info}\PY{l+s+s1}{\PYZsq{}}\PY{p}{)}
         
         \PY{n+nb}{print}\PY{p}{(}\PY{n}{header\PYZus{}row}\PY{p}{)}
\end{Verbatim}


    \begin{Verbatim}[commandchars=\\\{\}]
['url', 'Name/Alias', 'Appearances', 'Current?', 'Gender', 'Probationary Introl', 'Full/Reserve Avengers Intro', 'Year', 'Years since joining', 'Honorary', 'Death1', 'Return1', 'Death2', 'Return2', 'Death3', 'Return3', 'Death4', 'Return4', 'Death5', 'Return5', 'Notes\textbackslash{}n', 'More info', 'More info']
['url', 'Name/Alias', 'Appearances', 'Current?', 'Gender', 'Probationary Introl', 'Full/Reserve Avengers Intro', 'Year', 'Years since joining', 'Honorary', 'Death1', 'Return1', 'Death2', 'Return2', 'Death3', 'Return3', 'Death4', 'Return4', 'Death5', 'Return5', 'Notes\textbackslash{}n', 'More info', 'More info', 'More info']

    \end{Verbatim}

    \begin{Verbatim}[commandchars=\\\{\}]
{\color{incolor}In [{\color{incolor}33}]:} \PY{n}{header\PYZus{}row}\PY{p}{[}\PY{l+m+mi}{0}\PY{p}{]} \PY{o}{=} \PY{l+s+s1}{\PYZsq{}}\PY{l+s+s1}{url}\PY{l+s+s1}{\PYZsq{}}
         
         \PY{n+nb}{print}\PY{p}{(}\PY{n}{header\PYZus{}row}\PY{p}{)}
\end{Verbatim}


    \begin{Verbatim}[commandchars=\\\{\}]
['url', 'Name/Alias', 'Appearances', 'Current?', 'Gender', 'Probationary Introl', 'Full/Reserve Avengers Intro', 'Year', 'Years since joining', 'Honorary', 'Death1', 'Return1', 'Death2', 'Return2', 'Death3', 'Return3', 'Death4', 'Return4', 'Death5', 'Return5', 'Notes']

    \end{Verbatim}

    \begin{Verbatim}[commandchars=\\\{\}]
{\color{incolor}In [{\color{incolor}36}]:} \PY{k+kn}{import} \PY{n+nn}{copy} 
         
         \PY{n}{header\PYZus{}tuple} \PY{o}{=} \PY{n+nb}{tuple}\PY{p}{(}\PY{n}{header\PYZus{}row}\PY{p}{)}
         \PY{n}{header\PYZus{}copy} \PY{o}{=} \PY{n}{copy}\PY{o}{.}\PY{n}{copy}\PY{p}{(}\PY{n}{header\PYZus{}row}\PY{p}{)}
         
         
         \PY{n+nb}{print}\PY{p}{(}\PY{n+nb}{type}\PY{p}{(}\PY{n}{header\PYZus{}copy}\PY{p}{)}\PY{p}{)}
         \PY{n+nb}{print}\PY{p}{(}\PY{n+nb}{type}\PY{p}{(}\PY{n}{header\PYZus{}tuple}\PY{p}{)}\PY{p}{)}
\end{Verbatim}


    \begin{Verbatim}[commandchars=\\\{\}]
<class 'list'>
<class 'tuple'>

    \end{Verbatim}

    \begin{Verbatim}[commandchars=\\\{\}]
{\color{incolor}In [{\color{incolor}51}]:} \PY{n+nb}{print}\PY{p}{(}\PY{n}{header\PYZus{}tuple}\PY{p}{)}
\end{Verbatim}


    \begin{Verbatim}[commandchars=\\\{\}]
('url', 'Name/Alias', 'Appearances', 'Current?', 'Gender', 'Probationary Introl', 'Full/Reserve Avengers Intro', 'Year', 'Years since joining', 'Honorary', 'Death1', 'Return1', 'Death2', 'Return2', 'Death3', 'Return3', 'Death4', 'Return4', 'Death5', 'Return5', 'Notes')

    \end{Verbatim}

    \hypertarget{parsing-row-data-9-points}{%
\subsection{6. Parsing Row Data (9
points)}\label{parsing-row-data-9-points}}

From the \texttt{lines} variable you set in part 3, set the sixth line
to the variable \texttt{line}. This should be the entry for \emph{Thor
Odinson}. From this entry, we will create a dictionary with information
from this line in the CSV file.

1. Create a dictionary record

Given the field definitions defined below, create a dictionary called
\texttt{record} from the string data in the \texttt{line} variable. The
record should have keys corresponding to the \texttt{name} variable and
type corresponding with the \texttt{type} variable. The \texttt{index}
variable gives the index position of field value when the comma
separated line is parsed into a list.

\begin{Shaded}
\begin{Highlighting}[]
\NormalTok{fields }\OperatorTok{=}\NormalTok{ [}
\NormalTok{    \{}\StringTok{'index'}\NormalTok{: }\DecValTok{0}\NormalTok{, }\StringTok{'name'}\NormalTok{: }\StringTok{'url'}\NormalTok{, }\StringTok{'type'}\NormalTok{: }\BuiltInTok{str}\NormalTok{\}, }
\NormalTok{    \{}\StringTok{'index'}\NormalTok{: }\DecValTok{1}\NormalTok{, }\StringTok{'name'}\NormalTok{: }\StringTok{'name_alias'}\NormalTok{, }\StringTok{'type'}\NormalTok{: }\BuiltInTok{str}\NormalTok{\}, }
\NormalTok{    \{}\StringTok{'index'}\NormalTok{: }\DecValTok{2}\NormalTok{, }\StringTok{'name'}\NormalTok{: }\StringTok{'appearances'}\NormalTok{, }\StringTok{'type'}\NormalTok{: }\BuiltInTok{int}\NormalTok{\}, }
\NormalTok{    \{}\StringTok{'index'}\NormalTok{: }\DecValTok{3}\NormalTok{, }\StringTok{'name'}\NormalTok{: }\StringTok{'is_current'}\NormalTok{, }\StringTok{'type'}\NormalTok{: }\BuiltInTok{bool}\NormalTok{\}, }
\NormalTok{    \{}\StringTok{'index'}\NormalTok{: }\DecValTok{4}\NormalTok{, }\StringTok{'name'}\NormalTok{: }\StringTok{'gender'}\NormalTok{, }\StringTok{'type'}\NormalTok{: }\BuiltInTok{str}\NormalTok{\}, }
\NormalTok{    \{}\StringTok{'index'}\NormalTok{: }\DecValTok{7}\NormalTok{, }\StringTok{'name'}\NormalTok{: }\StringTok{'year'}\NormalTok{, }\StringTok{'type'}\NormalTok{: }\BuiltInTok{int}\NormalTok{\}, }
\NormalTok{    \{}\StringTok{'index'}\NormalTok{: }\DecValTok{8}\NormalTok{, }\StringTok{'name'}\NormalTok{: }\StringTok{'years_since_joining'}\NormalTok{, }\StringTok{'type'}\NormalTok{: }\BuiltInTok{int}\NormalTok{\}, }
\NormalTok{]}
\end{Highlighting}
\end{Shaded}

2. Since this is from an older dataset, the
\texttt{years\_since\_joining} value is no longer accurate. Update the
\texttt{years\_since\_joining} using the current year.

3. Using the \texttt{name\_alias} field, add two new names to the record
called \texttt{first\_name} and \texttt{last\_name}.

    \begin{Verbatim}[commandchars=\\\{\}]
{\color{incolor}In [{\color{incolor}46}]:} \PY{n}{line} \PY{o}{=} \PY{n}{lines}\PY{p}{[}\PY{l+m+mi}{5}\PY{p}{]}
         \PY{n}{values} \PY{o}{=} \PY{n}{line}\PY{o}{.}\PY{n}{split}\PY{p}{(}\PY{l+s+s1}{\PYZsq{}}\PY{l+s+s1}{,}\PY{l+s+s1}{\PYZsq{}}\PY{p}{)}
         \PY{n}{keys} \PY{o}{=} \PY{n}{header\PYZus{}row}
         
         \PY{n+nb}{print}\PY{p}{(}\PY{n}{thor}\PY{p}{)}
         \PY{n+nb}{print}\PY{p}{(}\PY{n}{header\PYZus{}row}\PY{p}{)}
         
         \PY{n}{record} \PY{o}{=} \PY{n+nb}{dict}\PY{p}{(}
             \PY{n}{URL}\PY{o}{=}\PY{n}{values}\PY{p}{[}\PY{l+m+mi}{0}\PY{p}{]}\PY{p}{,}
             \PY{n}{name\PYZus{}alias}\PY{o}{=}\PY{n}{values}\PY{p}{[}\PY{l+m+mi}{1}\PY{p}{]}\PY{p}{,}
             \PY{n}{appearances}\PY{o}{=}\PY{n+nb}{int}\PY{p}{(}\PY{n}{values}\PY{p}{[}\PY{l+m+mi}{2}\PY{p}{]}\PY{p}{)}\PY{p}{,}
             \PY{n}{is\PYZus{}current}\PY{o}{=}\PY{p}{(}\PY{n}{values}\PY{p}{[}\PY{l+m+mi}{3}\PY{p}{]}\PY{o}{.}\PY{n}{upper}\PY{p}{(}\PY{p}{)} \PY{o}{==} \PY{l+s+s1}{\PYZsq{}}\PY{l+s+s1}{YES}\PY{l+s+s1}{\PYZsq{}}\PY{p}{)}\PY{p}{,}
             \PY{n}{gender}\PY{o}{=}\PY{n}{values}\PY{p}{[}\PY{l+m+mi}{4}\PY{p}{]}\PY{p}{,}
             \PY{n}{year}\PY{o}{=}\PY{n+nb}{int}\PY{p}{(}\PY{n}{values}\PY{p}{[}\PY{l+m+mi}{7}\PY{p}{]}\PY{p}{)}\PY{p}{,}
             \PY{n}{years\PYZus{}since\PYZus{}joining}\PY{o}{=}\PY{n+nb}{int}\PY{p}{(}\PY{n}{values}\PY{p}{[}\PY{l+m+mi}{8}\PY{p}{]}\PY{p}{)}\PY{p}{)}
             
         \PY{n+nb}{print}\PY{p}{(}\PY{n}{record}\PY{p}{)}
\end{Verbatim}


    \begin{Verbatim}[commandchars=\\\{\}]
['http://marvel.wikia.com/Thor\_Odinson\_(Earth-616)', 'Thor Odinson', '2402', 'YES', 'MALE', '', 'Sep-63', '1963', '52', 'Full', 'YES', 'YES', 'YES', 'NO', '', '', '', '', '', '', "Dies in Fear Itself brought back because that's kind of the whole point. Second death in Time Runs Out has not yet returned\textbackslash{}n"]
['url', 'Name/Alias', 'Appearances', 'Current?', 'Gender', 'Probationary Introl', 'Full/Reserve Avengers Intro', 'Year', 'Years since joining', 'Honorary', 'Death1', 'Return1', 'Death2', 'Return2', 'Death3', 'Return3', 'Death4', 'Return4', 'Death5', 'Return5', 'Notes']
\{'URL': 'http://marvel.wikia.com/Thor\_Odinson\_(Earth-616)', 'name\_alias': 'Thor Odinson', 'appearances': 2402, 'is\_current': True, 'gender': 'MALE', 'year': 1963, 'years\_since\_joining': 52\}

    \end{Verbatim}

    \begin{Verbatim}[commandchars=\\\{\}]
{\color{incolor}In [{\color{incolor}49}]:} \PY{n}{record}\PY{p}{[}\PY{l+s+s1}{\PYZsq{}}\PY{l+s+s1}{years\PYZus{}since\PYZus{}joining}\PY{l+s+s1}{\PYZsq{}}\PY{p}{]} \PY{o}{=} \PY{l+m+mi}{2018} \PY{o}{\PYZhy{}} \PY{n}{record}\PY{p}{[}\PY{l+s+s1}{\PYZsq{}}\PY{l+s+s1}{year}\PY{l+s+s1}{\PYZsq{}}\PY{p}{]}
         \PY{p}{(}\PY{n}{record}\PY{p}{)}
\end{Verbatim}


\begin{Verbatim}[commandchars=\\\{\}]
{\color{outcolor}Out[{\color{outcolor}49}]:} \{'URL': 'http://marvel.wikia.com/Thor\_Odinson\_(Earth-616)',
          'name\_alias': 'Thor Odinson',
          'appearances': 2402,
          'is\_current': True,
          'gender': 'MALE',
          'year': 1963,
          'years\_since\_joining': 55\}
\end{Verbatim}
            
    \begin{Verbatim}[commandchars=\\\{\}]
{\color{incolor}In [{\color{incolor}50}]:} \PY{n}{record}\PY{p}{[}\PY{l+s+s1}{\PYZsq{}}\PY{l+s+s1}{first\PYZus{}name}\PY{l+s+s1}{\PYZsq{}}\PY{p}{]}\PY{p}{,} \PY{n}{record}\PY{p}{[}\PY{l+s+s1}{\PYZsq{}}\PY{l+s+s1}{last\PYZus{}name}\PY{l+s+s1}{\PYZsq{}}\PY{p}{]} \PY{o}{=} \PY{n}{record}\PY{p}{[}\PY{l+s+s1}{\PYZsq{}}\PY{l+s+s1}{name\PYZus{}alias}\PY{l+s+s1}{\PYZsq{}}\PY{p}{]}\PY{o}{.}\PY{n}{split}\PY{p}{(}\PY{l+s+s1}{\PYZsq{}}\PY{l+s+s1}{ }\PY{l+s+s1}{\PYZsq{}}\PY{p}{)}
         \PY{n+nb}{print}\PY{p}{(}\PY{n}{record}\PY{p}{)}
\end{Verbatim}


    \begin{Verbatim}[commandchars=\\\{\}]
\{'URL': 'http://marvel.wikia.com/Thor\_Odinson\_(Earth-616)', 'name\_alias': 'Thor Odinson', 'appearances': 2402, 'is\_current': True, 'gender': 'MALE', 'year': 1963, 'years\_since\_joining': 55, 'first\_name': 'Thor', 'last\_name': 'Odinson'\}

    \end{Verbatim}


    % Add a bibliography block to the postdoc
    
    
    
    \end{document}
